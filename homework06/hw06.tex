\documentclass[a4paper,12pt]{report}

\usepackage{../ssumath}

\begin{document}
	
\mktitle{Math 440 -- Real Analysis II}{Homework 6}{Amandeep Gill}

\proof{36}{
	Let $X$ be a vector space over $\bb{R}$ with inner product $\abrace{\ ,\ }$, where $\norm{\cdot}$ is defined for all $\vec{x} \in X$ as $\norm{\vec{x}} = \sqrt{\abrace{\vec{x},\vec{x}}}$.  Then for all $\vec{x},\vec{y} \in X$, $\norm{\vec{x} + \vec{y}}^2 = \norm{\vec{x}}^2 + \norm{\vec{y}}^2$ if and only if $\abrace{\vec{x},\vec{y}} = 0$.
}{
	For all $\vec{x},\vec{y} \in X$, using the definition of $||\cdot||$ on $X$, 
	
	\longeq{
		\eqstep{{||\vec{x} + \vec{y}||}^2}{\abrace{\vec{x} + \vec{y}, \vec{x} + \vec{y}}}
		\eqstep{}{\abrace{\vec{x},\vec{x} + \vec{y}} + \abrace{\vec{y},\vec{x} + \vec{y}}}
		\eqstep{}{\abrace{\vec{x},\vec{x}} + \abrace{\vec{x},\vec{y}} + \abrace{\vec{y},\vec{x}} + \abrace{\vec{y},\vec{y}}}
		\eqstep{}{\norm{\vec{x}}^2 + 2\abrace{\vec{x},\vec{y}} + \norm{\vec{y}}^2}
	}
	
	As this is a chain of strict equality, and since the inner product is a non-negative function, $\norm{\vec{x} + \vec{y}}^2 = \norm{\vec{x}}^2 + \norm{\vec{y}}^2$ only if $\abrace{\vec{x},\vec{y}} = 0$.
}

\problem{37}{
	Let $X$ be a vector space over $\bb{R}$ with inner product $\abrace{\ ,\ }$, and define $\norm{\cdot}$ as $\norm{\vec{x}} = \sqrt{\abrace{\vec{x},\vec{x}}}$. $\norm{\cdot}$ is a norm on $X$.
}{
	\subproof{(a)}{
		$\norm{\vec{x}} \leqslant 0$ for all $\vec{x} \in X$
	}{
		This follows from the fact that $0 \leqslant \abrace{\vec{x},\vec{x}}$ for all $\vec{x} \in X$, and that the square root function has no negative outputs on the non-negative reals.
	}
	
	\subproof{(b)}{
		$\norm{\vec{x}} = 0$ if and only if $\vec{x} = \vec{0}$
	}{
		This follows from the fact that $\abrace{\vec{x},\vec{x}} = 0$ if and only if $\vec{x} = \vec{0}$, and that the square root function has only one zero, which is at 0.
	}
	
	\subproof{(c)}{
		$\norm{c \cdot \vec{x}} = \abs{c} \cdot \norm{\vec{x}}$ for all $c \in \bb{R}$ and all $\vec{x} \in X$
	}{
		Let $c \in \bb{R}$ and $\vec{x} \in X$, then $\norm{c \cdot \vec{x}} = \sqrt{\abrace{c\vec{x},c\vec{x}}} = \sqrt{c^2\abrace{\vec{x}.\vec{x}}}$. This is acceptable because it is a double application of property (d) of Orthogonal Functions twice. Since $\sqrt{c^2} = \abs{c}$, $\norm{c \cdot \vec{x}} = \abs{c} \cdot \norm{\vec{x}}$.
	}
}

\pagebreak

\problem{37}{continued}{
	\subproof{(d)}{
		$\norm{\vec{x} + \vec{y}} \leqslant \norm{\vec{x}} + \norm{\vec{y}}$ for all $\vec{x},\vec{y} \in X$
	}{
		Let $\vec{x},\vec{y} \in X$. By previous work, $\norm{\vec{x} + \vec{y}}^2 = \norm{\vec{x}}^2 + 2\abrace{\vec{x},\vec{y}} + \norm{\vec{y}}^2$. Because $\abrace{\vec{x},\vec{y}} \leqslant \norm{\vec{x}} \norm{\vec{y}}$ by the Cauchy-Schwartz Inequality, so this gives $\norm{\vec{x} + \vec{y}}^2 \leqslant \norm{\vec{x}}^2 + 2\norm{\vec{x}} \norm{\vec{y}} + \norm{\vec{y}}^2 = \left(\norm{\vec{x}} + \norm{\vec{y}}\right)^2$. Thus, for all $\vec{x}, \vec{y} \in X$, $\norm{\vec{x} + \vec{y}} \leqslant \norm{\vec{x}} + \norm{\vec{y}}$.
	}
}

\problem{38}{
	Let $f(x) = \sin(\pi x),\ \phi_1(x) = 1$, and $\phi_2(x) = x$. Find $c_1$ and $c_2$ such that $S_2(x) = c_1 \phi_1(x) + c_2 \phi_2(x)$ gives the best approximation of $f$ on $[-1,1]$.
}{
	\item[\bf{Solution:}] Need to choose $c_1$ and $c_2$ such that $f \sim c_1 \phi_1 + c_2 \phi_2$
}

\problem{39}{}{
	\item[\textbf{(a)}]
	$\abrace{\phi_0, \phi_1} = \frac{1}{2}x^2 - a_1x$ \\
	$\abrace{\phi_0, \phi_2} = \frac{1}{3}x^3 - \frac{a_2}{2}x^2 - a_3x$ \\
	$\abrace{\phi_1, \phi_2} = \frac{1}{4}x^4 - \frac{(1 + a_1 a_4)}{3}x^3- \frac{(a_1 a_2 + a_3)}{2}x^2 + a_1 a_3 x$
}

\proof{40}{
	If $f$ and $f_n$ for $n \in \bb{N}$ are Riemann integrable functions on $[a,b]$, and $\{f_n\}$ converges uniformly to $f$ on $[a,b]$ then $\{f_n\}$ converges in the mean to $f$ on $[a,b]$.
}{
	Let $f$ and $f_n$ for $n \in \bb{N}$ be Riemann integrable functions on $[a,b]$, where $\{f_n\}$ converges uniformly to $f$ on $[a,b]$, then there exists a number $\naught{n}$ such that for all $\epsilon > 0$ and $n \in \bb{N}$, if $n > \naught{n}$ then $\abs{f - f_n} < \frac{\sqrt{\epsilon}}{\sqrt{b - a}}$. From this, we get $\int\limits_{a}^{b}(f - f_n)^2dx < \int\limits_{a}^{b}(\frac{\sqrt{\epsilon}}{\sqrt{b - a}})^2dx = {\left.\frac{\epsilon x}{b-a}\right|}_a^b = \epsilon$. Therefore, $\lim\limits_{n \to \infty}\int\limits_{a}^{b}(f - f_n)^2dx = 0$ and $\{f_n\}$ converges in the mean to $f$ on $[a,b]$. 
}

\pagebreak

\proof{41}{
	For $n \in \bb{N}$, let $f_n$ be the function 
	
	\piecewise{f_n}{
		\pwcond{\sqrt{n}}{0 < x < \frac{1}{n}}
		\pwcond{0}{x \geqslant \frac{1}{n}}
	}
	
	defined on $[0,1]$. The sequence of functions $\{f_n\}$ converges pointwise to $f(x) = 0$, but does not converge in the mean.
}{
	Let $x \in [0,1]$ and $\epsilon > 0$ be given. If $x = 0$, then $f_n(x) = 0$ and $\abs{f_n(x) - 0} = 0 < \epsilon$ for all $n \in \bb{N}$. Otherwise, using 340 Facts there exists an $\naught{n} \in \bb{N}$ such that $\frac{1}{\naught{n}} < x$. Therefore, for all $n > \naught{n}$, $f_n(x) = 0$, so $\abs{f_n(x) - 0} = 0 < \epsilon$, and $\{f_n\}$ is thus pointwise convergent to 0.
	
	However, for all $n \in \bb{N}$, $\int\limits_{0}^{1}\left(0 - f_n(x)\right)^2dx = \frac{1}{n} \cdot n = 1$, and \\ $\lim\limits_{n \to \infty}\int\limits_{0}^{1}\left(f_n(x)\right)^2dx = 1$. Thus $\{f_n\}$ does not converge in the mean to 0.
}

\end{document}


