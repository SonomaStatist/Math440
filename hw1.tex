\documentclass[a4paper, 12pt]{report}

\usepackage[a4paper]{geometry}
\geometry{a4paper,tmargin=1in,bmargin=1in}

\usepackage{amsmath}
\usepackage{amssymb}
\usepackage{amsfonts}

%Custom command to shorten mathbb for basic number sets
\newcommand{\bb}[1]{\mathbb{#1}}

%Custom command for defining problem{#}{theorem}proof:{text}blacksquare
\newcommand{\proof}[3]{
	\begin{enumerate}
		\item[\bf{Problem #1}] #2
		\begin{enumerate}
			\item[\bf{Proof:}]
			#3
		\end{enumerate}
	\end{enumerate}
	\begin{flushright}
		$\blacksquare$
	\end{flushright}
}

%Custom command for defining problem{#}{text}
\newcommand{\problem}[2]{
	\begin{enumerate}
		\item[\bf{Problem #1}] 
		#2
	\end{enumerate}
}

%Custom command for defining {item}{theorem}proof:{text}blacksquare
%does require \b{enum} ... \e{enum} or \problem
\newcommand{\subproof}[3]{
	\item[#1] #2
	\item[\bf{Proof:}]
	#3 
	\begin{flushright}
		$\blacksquare$
	\end{flushright}
}

\begin{document}

\title{Math 322 -- Linear Algebra \\ \vspace{7px} \large{Homework 1}}
\author{Amandeep Gill}
\maketitle

\proof{1}{
	Let $V = \{(a_1,a_2) : a_1,a_2 \in \bb{R}\}$ with the operations component-wise addition and 
	$ \forall c \in \bb{R} ,$
	\[ c (a_1,a_2) = 
		\left\{
		\begin{array}{ll}
			(0,0) & \text{if } c = 0 \\
			(c a_1, \frac{a_2}{c}) & \text{if } c \ne 0
		\end{array}
		\right.
	\]
	Show whether $V$ is a vector space over $\bb{R}$
}{
	$V$ is not a vector space since it fails the distributive property of vector over scalars \\
	$(1 + 1)\vec{a} = ((1 + 1)a_1, \frac{a_2}{1 + 1}) = (2a_1,\frac{a_2}{2}) \ne (a_1 + a_1,a_2 + a_2) = (2a_1,2a_2)$
}

\proof{2}{
	Show that if $W \subseteq V$ and $V$ is a vector space over some field $F$, $W$ is a vector space if and only if $span(W) = W$
}{
	\item[$(\Rightarrow)$] 
		Let $W$ be a subspace of $V$, then by the definition of vector spaces $\forall \vec{w_1}, \vec{w_2} \in W$ and $\forall c \in F,\ c \vec{w_1} + \vec{w_2} \in W$. Since $W \subseteq span(W)$ and $span(W) \subseteq W$ as the span consists of all linear combination of elements of $W$, $span(W) = W$. \\
	\item[$(\Leftarrow)$]
		Let $span(W) = W$. Since $W \subseteq V$ and because $span(W)$ denotes all possible linear combinations of elements from $W,\ \forall \vec{w_1}, \vec{w_2} \in W$ and $\forall c \in F,\ c \vec{w_1} + \vec{w_2} \in W$. Therefore $W$ is a vector space.
}

\proof{3}{
	Let $V$ be a vector space over some field $F$ such that $dim(V) = n$, and let $S \subseteq V$ such that $span(S) = V$, then $\exists \beta \subseteq S$ such that $span(\beta) = V,\ |\beta| = n$, and $|S| \geqslant n$.
}{
	Since $V$ is a vector space, $V$ has a basis $\beta_1$ such that $span(\beta_1) = V$ and $|\beta| = n$. If $\beta_1 \subseteq S$, then the hypothesis holds. If $\beta_1 \not\subseteq S$, then $\beta_1 \subseteq span(S)$. This means that $\forall b_i \in \beta,\ \exists s_1,s_2,\ldots,s_m \in S$ and $c_1,c_2,\ldots,c_m \in F$ such that $b_i = c_1 s_1 + c_2 s_2 + \cdots + c_m s_m$. Let $B$ be the set containing every $s_i$ from $S$ where the scalar multiplier is not the 0 in $F$, then $|B| \leqslant |\beta| \cdot m$ and $span(B) = S$. Because $B$ is finite set that spans $S,\ \exists \beta \subseteq B$ such that $\beta \subseteq S$ and $span(\beta) = V$, and $|\beta| = n$. It also follows that $|S| \geqslant |\beta| = n$.
}

\end{document}