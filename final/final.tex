\documentclass[a4paper,12pt]{report}

\usepackage{../ssumath}

\begin{document}
	
\mktitle{Math 440 -- Real Analysis II}{Final Exam}{Amandeep Gill}

\problem{1}{
	Determine which of the following series are divergent, conditionally convergent, or absolutely convergent.
}{
	\subproof{(a)}{
		$\sum\limits_{k=1}^{\infty} \frac{(-1)^k k}{2k^2 - 1}$ converges conditionally.
	}{
		Let $b_k = \frac{k}{2k^2 - 1}$. Since $b_{k+1} = \frac{k + 1}{2(k+1)^2 - 1} = \frac{k + 1}{2k^2 + 4k + 1} < b_k$ for all $k \in \bb{K}$ and $\lim\limits_{k \to \infty}b_k = 0$, by the Alternating Series Test $\sum\limits_{k=1}^{\infty} \frac{(-1)^k k}{2k^2 - 1}$ converges conditionally.
	}
	
	\subproof{(b)}{
		$\sum\limits_{k=0}^{\infty} \frac{(-1)^{a_k}}{k^2 + 1}$ where $a_k = \lfloor\frac{k}{2}\rfloor$ converges absolutely.
	}{
		Let $b_k = \abs{\frac{(-1)^{a_k}}{k^2 + 1}} = \frac{1}{k^2 + 1}$. For all $k \in \bb{N}$, $b_k < \frac{1}{k^2}$, therefore since $\frac{1}{k^2}$ is a convergent P-series $\sum\limits_{k=1}^{\infty}b_k$ converges. Thus $\sum\limits_{k=0}^{\infty} \frac{(-1)^{a_k}}{k^2 + 1}$ converges absolutely.
	}
}

%\pagebreak

\problem{2}{
	Let $\cbrace{q_k}_{k\in\bb{N}}$ be an enumeration of the rational numbers on $[0,1]$, and for each $k \in \bb{N}$ define $C_k = \cbrace{q_1, q_2, \ldots, q_k}$. Let $f : [0,1] \to \bb{R}$ be defined by $f(x) = \sum\limits_{k=1}^{\infty} 2^{-k} \chi_{C_k}$.
}{
	\subproof{(a)}{
		The series defining $f$ converges uniformly on $[0,1]$. 
	}{
		Let $f$ be defined piecewise as
		
		\piecewise{f(x)}{
			\pwcond{0}{x \in [0,1] \backslash \bb{Q}}
			\pwcond{2^{1-i}}{x = q_i \text{ for some } q_i \in \cbrace{q_k}_{k \in \bb{N}}}
		}
		
		Let $\epsilon > 0$ be given, $f_n(x) = \sum\limits_{k=1}^{n} 2^{-k} \chi_{C_k}$, and let $m, \naught{n} \in \bb{N}$ such that $m > \naught{n}$ and $2^{-\naught{n}} < \epsilon$. If $x \in [0,1] \backslash \bb{Q}$, then $f_m(x)=0$ and $\abs{f_m(x) - f(x)} = 0 < \epsilon$. Otherwise, $x = q_i$ for some $i \in \bb{N}$. 
		\begin{enumerate}
		\case{$i > m$}:\\
			$f_m(x) = 0$ and $\abs{f_m(x) - f(x)} = 2^{1-i} < 2^{-\naught{n}} < \epsilon$.
		\case{$i \leqslant m$}:\\
			$f_m(x) = 2^{1-i} - 2^{-m}$ and $\abs{f_m(x) - f(x)} = 2^{-m} < 2^{-\naught{n}} < \epsilon$.
		\end{enumerate}
		
		Thus $f_n$ converges to $f$ uniformly on $[0,1]$.
	}
	
	\subsoln{(b)}{
		Calculate $f(q_1)$, $f(q_2)$, $f(q_3)$, and $f(\sqrt{2}/2)$
	}{ $\ $
		\begin{enumerate}
		\item[$f(q_1)$] = $1$
		\item[$f(q_2)$] = $\frac{1}{2}$
		\item[$f(q_3)$] = $\frac{1}{4}$
		\item[$f(\sqrt{2}/2)$] = $0$
		\end{enumerate}
	}
	
	%\pagebreak
	
	\subproof{(c)}{
		$f$ is Riemann integrable on $[0,1]$ and $\int\limits_{0}^{1}f dx = 0$
	}{
		For each $n \in \bb{N}$ let $\partn_n = \cbrace{x_0,x_1,\ldots,x_{2^n}}$ partition $[0,1]$ where $x_i = \frac{i}{2^n}$. For each $n$, $\mathcal{U}(\partn_n,f) \leqslant \sum\limits_{k=1}^{2^n} 2^{1-k} \frac{1}{2^{n}} = 2^{1-n} - 2^{1-n-2^n}$ as the sup of an interval in the partition must be $f(q_1)$ or less, which means that the sup of one of the remaining intervals must be $f(q_2)$ or less, and so on up to $f(q_{2^n})$ in the final interval. Thus by Lemma 6.1.3 \[\inf\mathcal{U}(\partn,f) \leqslant \lim\limits_{n\to\infty} \mathcal{U}(\partn_n,f) \leqslant \lim\limits_{n\to\infty}2^{1-n} - 2^{1-n-2^n} = 0\] $f$ is non-negative, so by Theorem 6.1.4, \[0 \leqslant \sup\mathcal{L}(\partn, f) \leqslant \inf\mathcal{U}(\partn,f) \leqslant 0\] Therefore $\int\limits_{0}^{1}fdx = 0$ by Defintion 6.1.5.
	}
	
	\subproof{(c)}{
		$f$ is Lebesgue integrable on $[0,1]$ and $\int\limits_{0}^{1}f d\lambda = 0$
	}{
		By Corollary 10.6.8 if the Riemann integral exists, then the Lebesgue integral exists and is equal to the Riemann integral. Thus $\int\limits_{0}^{1}f d\lambda = 0$.
	}
}

%\pagebreak

\problem{3}{
	Let $A$ be a measurable subset of $\bb{R}$, and let $\mathcal{L}(A)$ denote the set of all Lebesgue integrable functions on $A$. Given $f,g \in \mathcal{L}(A)$ and $c \in \bb{R}$, define $(f+g)(x) = f(x) + g(x)$ and $(cf)(x) = cf(x)$ for all $x \in \bb{R}$
}{
	\subproof{(a)}{
		$\mathcal{L}(A)$ forms a vector space over $\bb{R}$.
	}{
		Let $f,g \in \mathcal{L}(A)$ and $c \in \bb{R}$, then $\int\limits_{A}(f + g) d\lambda = \int\limits_{A}f d\lambda + \int\limits_{A}f d\lambda$ by Theorem 10.7.8(a), as both $f$ and $g$ are Lebesgue integrable functions by assumption, so $(f+g)(x)$ is Lebesgue integrable and $\mathcal{L}(A)$ is closed under function addition. Similarly by Theorem 10.7.8(a), $\int\limits_{A}cf d\lambda = c \int\limits_{A}f d\lambda$, hence $(cf)(x)$ is Lebesgue integrable and $\mathcal{L}(A)$ is closed under scalar multiplication. Because the eight vector space axioms hold trivially, $\mathcal{L}(A)$ is a vector space over $\bb{R}$.
	}
	
	\subproof{(c)}{
		If $\norm{\cdot} : \mathcal{L}(A) \to \bb{R}$ is defined as $\norm{f} = \int\limits_{A} \abs{f} d\lambda$, then $\norm{\cdot}$ is not a norm on $\mathcal{L}(A)$
	}{
		Let $A = [0,1]$ and $f(x) = \chi_E$ where $E = [0,1] \backslash \bb{Q}$. Then $\norm{f} = \int\limits_{A} \abs{f} d\lambda = \int\limits_{A} \chi_E d\lambda$. By Theorem 10.6.10, $\int\limits_{A} \chi_E d\lambda = \int\limits_{E} 1 d\lambda + \int\limits_{E^c} 0 d\lambda$. Let $x_n \in E$ such that $\bigcup\limits_{n \in \bb{N}} \cbrace{x_n} = E$, and let $\epsilon > 0$ be given. Then for each $I_n = (x_n - \frac{\epsilon}{2^{n+1}}, x_n + \frac{\epsilon}{2^{n+1}})$, $E \subset \bigcup\limits_{n\in\bb{N}}I_n$ and $\sum\limits_{n \in \bb{N}} \lambda(I_n) = \epsilon$. Thus, by Theorem 6.1.11, $\lambda(E) = 0$ and $\int\limits_{E} 1 d\lambda + \int\limits_{E^c} 0 d\lambda = 1\lambda(E) + 0\lambda(E^c) = 0$. Since $f \not= 0$  for all $x \in [0,1]$, $\norm{\cdot}$ does not satisfy the property that $\norm{f} = 0$ if and only if $f = 0$, and therefore $\norm{\cdot}$ is not a norm on $\mathcal{L}(A)$.
	}
}

%\pagebreak

\problem{(4)}{
	Let $a \in \bb{R}$ such that $0 < a < \frac{1}{2}$, and let $C_0 = [0,1]$ be the first step in the generalized Cantor set with $C_n$ comprised of $2^n$ disjoint intervals of length $a^n$ such that $C_n \subset C_{n-1}$, and define \[C_a = C_0 \cap C_1 \cap C_2 \cap \cdots\] 
}{
	\subproof{(a)}{
		If $a \in \bb{R}$ with $0 < a < \frac{1}{2}$ then the set $C_a$ is measurable with measure 0.
	}{
		For each $n \in \bb{N}$, $C_n$ is the union of $2^n$ disjoint intervals of length $a^n$, so by Theorem 10.4.5(b) $\lambda(C_n) = 2^n a^n = (2a)^n$. Let $\cbrace{c_n}_{n \in \bb{N}}$ be a sequence of real numbers such that $c_n = \lambda(C_n) = (2a)^n$. Because the number of intervals is countable for all $n$, Theorem 10.4.5(b) holds for all $n$, and since $a < \frac{1}{2}$, $2a < 1$ and $c_n$ is a monotonically decreasing sequence bounded by $[0,1]$ and thus $\lim\limits_{n \to \infty} c_n = 0$. As $C_a \subset C_n$, for each open cover $U \supset C_n$ so $C_a \subset U$ and so $\lambda^*(C_a) \leqslant \lambda^*(C_n)$ by Definition 10.3.1. By Theorem 10.3.4(a) $\lambda(C_n) = \lambda^*(C_n)$, hence $\lambda^*(C_a) \leqslant c_n$ for all $n$. Therefore $0 \leqslant \lambda_*(C_a) \leqslant \lambda^*(C_a) \leqslant \lim\limits_{n \to \infty} c_n = 0$, and by Definition 10.3.4(a) $C_a$ is measurable with $\lambda(C_a) = 0$. 
	}
	
	\subproof{(b)}{
		$\frac{\log 2}{\log(1/a)}$ is an upper bound on the Hausdorff dimension of $C_a$.
	}{
		Let $A_n$ be defined as a ball in $\bb{R}$ such that $\abs{A_n} = a^n$ is a $\delta_n$ cover for each $n \in \bb{N}$. The number of $A_n$ balls needed to cover $C_a$ is then $N = 2^n$. By Definition D.2.4, $\mathcal{H}_{\delta_n}^s(C_a) \leqslant \sum\limits_{k = 1}^{N} \abs{A_n}^s = 2^n a^{sn}$. Therefore, as $\mathcal{H}^s(C_a) = \lim\limits_{\delta \to 0^+} \mathcal{H}_\delta^s(C_a) \leqslant \lim\limits_{n \to \infty} 2^n a^{sn}$. Thus $s = \frac{\log 2}{\log(1/a)}$ is an upper bound on $\dim_H(C_a)$.
	}
}

%\pagebreak

\proof{5}{
	The Hausdorff dimension of any countable subset of $\bb{R}^d$ is zero.
}{
	Let $\epsilon > 0$ be given, and let $X = \cbrace{x_1, x_2, \ldots, x_k}$ for some $k \in \bb{N}$ be a subset of $\bb{R}^d$ with a $\delta_\epsilon$-cover defined as $\bigcup\limits_{n = 1}^{k}C_n$ where $x_n \in C_n$ and $\abs{C_n} = \frac{\epsilon}{2^n}$. By Definition D.2.4, $\mathcal{H}_{\delta_\epsilon}^s \leqslant \sum\limits_{n=1}^{k} (\frac{\epsilon}{2^n})^s = \left(\epsilon - \frac{\epsilon}{2^k}\right)^s$. Thus $\lim\limits_{\delta_\epsilon \to 0^+} \mathcal{H}_{\delta_\epsilon}^s \leqslant 0^s = 0$ for all $s > 0$ and therefore as $k$ is arbitrary, $\dim_H(X) = 0$ for all $k \in \bb{N}$.
} 

\proof{6}{
	Let $\cbrace{E_k}_{k \in \bb{N}}$ be a sequence of subsets of $\bb{R}$. If there exists a sequence $\cbrace{U_k}_{k \in \bb{N}}$ of pairwise disjoint open sets such that $E_k \subset U_k$ for all $k \in \bb{N}$, then \[\lambda^*\left(\bigcup\limits_{k = 1}^{\infty}E_k\right) = \sum\limits_{k = 1}^{\infty}\lambda^*(E_k)\]
}{
	Let $\cbrace{U_k}_{k \in \bb{N}}$ be a sequence of disjoint open sets such that $E_k \subset U_k$ for all $k$, and let $U$ be the union of all $U_k$ and $E$ be the union of all $E_K$. By Theorem 10.4.4(a), $\lambda^*(E) \leqslant \sum\limits_{k=1}^{\infty} \lambda^*(E_k)$. 
	
	\begin{enumerate}
	\case{$\lambda^*(E) = \infty$} 
		From the above, $\infty \leqslant  \sum\limits_{k=1}^{\infty} \lambda^*(E_k)$ and so $\sum\limits_{k=1}^{\infty} \lambda^*(E_k) = \lambda^*(E)$. 
		
	\case{$\lambda^*(E) < \infty$}
		Because $U_i \cap U_j = \emptyset$ for all $i,j \in \bb{N}$ where $i \not= j$, $E_i$ and $E_j$ are similarly disjoint. Thus, given $\epsilon > 0$, by Definition 10.3.1 there exists an open set $A$ where $E \subset A$, $m(A) < \infty$, and for each $k \in \bb{N}$, let $V_k \subset A$ be an open set such that $E_k \subset V_k \subset U_k$ and $\lambda^*(E_k) = m(V_k) - \frac{\epsilon}{2^k}$.  Given that $V_k$ is an open set $m(V_k) = \lambda(V_k)$ by Theorem 10.3.2(Ex2), and so from Theorem 10.4.5(b) and the fact that $V_k$ are pairwise disjoint, $m(V) = \sum\limits_{k=1}^{\infty}m(V_k)$ where $V$ is the union of $V_k$ for all $k$. $V$ is a subset of $A$, so Theorem 10.2.4 gives $m(V) \leqslant m(A)$. Hence \[\sum\limits_{k=1}^{\infty} \lambda^*(E_k) = \sum\limits_{k=1}^{\infty}\left(m(V_k) - \frac{\epsilon}{2^k}\right) \leqslant m(V) < \infty\] 
	\end{enumerate}
	
	Therefore $\sum\limits_{k=1}^{\infty} \lambda^*(E_k) = m(V) - \epsilon = m(V)$, since $\epsilon$ is arbitrary. Similarly from Definition 10.3.1, for each $\epsilon > 0$ there exists an open set $B$ such that $E \subset B \subset V$ and $\lambda^*(E) = m(B) - \epsilon$, and again since $\epsilon$ is arbitrary, $\lambda^*(E) = m(B)$. Thus
	
	\longeq{
		\eqstep{\sum\limits_{k=1}^{\infty}\lambda^*(E_k) + \lambda^*(E)}{m(V) + m(B)}
		\leqstep{\lambda^*(E) + \lambda^*(E)}{m(V) + m(B)}
		\leqstep{\lambda^*(E) + \lambda^*(E)}{\sum\limits_{k=1}^{\infty}\lambda^*(E_k) + \lambda^*(E)}
		\leqstep{\lambda^*(E)}{\sum\limits_{k=1}^{\infty}\lambda^*(E_k)}
	}
	
	Therefore $\lambda^*(E) = \sum\limits_{k=1}^{\infty}\lambda^*(E_k)$.
}

\problem{7}{
	Let $\gamma$ be given such that $0 < \gamma < 1$, and let the following be the generalized construction of the Cantor set with $C_0 = [0,1]$ and $C_{k+1}$ generated by removing the center open interval of length $\gamma/3^{k+1}$ from each of the $2^k$ interval of $C_k$. Define $C_\gamma$ as \[C_\gamma = C_0 \cap C_1 \cap C_2 \cap \cdots\]
}{
	\subproof{(a)}{
		$C_\gamma$ has positive measure with no intervals.
	}{
		An alternate definition for $C_\gamma$ can be defined as the set $[0,1] \backslash A_\gamma$, where $A_\gamma = \bigcup\limits_{n=1}^{\infty} A_n$ such that $A_i,A_j$ are pairwise disjoint when $i \not= j$ and each $A_n$ is the union of $2^{n-1}$ disjoint open intervals of length $\gamma/3^n$. For each $n$, $m(A_n) = \sum\limits_{i=1}^{2^{n-1}} \frac{\gamma}{3^n}$ using Definition 10.2.2, and so from Theorem 10.3.2(Ex2) $\lambda(A_n) = \gamma\left(\frac{2^{n-1}}{3^n}\right)$. Because $A_\gamma$ is a countable union of disjoint open sets, $\lambda(A_\gamma) = \sum\limits_{n=1}^{\infty}\lambda(A_n) = \gamma$ by Theorem 10.4.5. Since $C_\gamma$ is complementary to $A_\gamma$ on $[0,1]$, $C_\gamma$ has measure $\lambda(C_\gamma) = 1 - \lambda(A_\gamma)$ by Corollary 10.4.3 and Theorem 10.3.7. Thus $\lambda(C_\gamma) = 1 - \gamma$. However, given $\epsilon > 0$ and $x \in C_\gamma$, let $I = [x - \epsilon, x + \epsilon]$. For some $n \in \bb{N}$, the length of each closed interval in $C_n$ is $l =  \frac{1-\gamma}{3^n} < \epsilon$. As $x \in C_n$, $x$ is also in some $I_n$ of length  $l$ with a gap in between $I_n$ and the adjacent intervals, therefore $I_n \subset I$ and $I$ contains all points that lie on the same interval as $x$, and as $\epsilon$ is arbitrary, $I = \cbrace{x}$ and there are no intervals in $C_\gamma$. 
	}
	
	\subproof{(b)}{
		If $f : [0,1] \to \bb{R}$ as $f(x) = \chi_{C_\gamma}(x)$, then $f$ is not Riemann integrable.
	}{
		Because there are no intervals on $C_\gamma$, each $x \in C_\gamma$ is a jump discontinuity of $f$. Hence the measure of the set of discontinuities of $f$ has measure $1-\gamma > 0$, and $f$ is not Riemann integrable by Theorem 6.1.13.
	}
	
	\subproof{(c)}{
		As defined in (b), $f$ is Lebesgue integrable.
	}{
		$f$ is a simple, non-negative function defined on a measurable set, so $\int\limits_{[0,1]}f d\lambda = \int\limits_{C_\gamma}1d\lambda + \int\limits_{A_\gamma}0d\lambda = (1-\gamma)$ by Theorem 10.6.10(b).
	}
}

\end{document}






















