\documentclass[a4paper,12pt]{report}

\usepackage{amsmath}
\usepackage{amssymb}
\usepackage{amsfonts}

\begin{document}

\title{Math 440 -- Real Analysis II \\ \vspace{7px} \large{Homework 1}}
\author{Amandeep Gill}
\maketitle

\begin{enumerate}

\item[\bf{Problem 1}] Find the limit inferior and limit superior of each of the following sequences \{$s_n$\}. 
	\begin{enumerate}
	\item{$s_n = 2 - \frac{1}{n}$} \\
		\[\varlimsup \{s_n\} = \varliminf \{s_n\} = \lim\limits_{n \to \infty}{s_n} = 2\]

	\item{$s_n = n \bmod{4}$} \\\\
		For all $k \in \mathbb{N}$ the sequence $\{s_k, s_{k+1}, s_{k+2}, \ldots\}$ contains only the values $\{0, 1, 2, 3\}$. Thus $\forall{k}$, $0 \leqslant s_k \leqslant 3$, so
		\[\varlimsup \{s_n\} = 3 \]
		\[\text{And}\]
		\[\varliminf \{s_n\} = 0 \] 

	\item{$s_n = \begin{cases}
		n & \text{if $n$ is even} \\
		\frac{1}{n} + \cos{\pi n} & \text{if $n$ is odd}
		\end{cases}$} \\
			
		For all $n \in \bf{N}$, $\frac{1}{n} + \cos{\pi n} \leq n$, so the $\sup\{s_k, s_{k+1}, s_{k+2}, \ldots\}$ is dominated by the even terms of $s_n$. Therefore
		\[\varlimsup \{s_n\} = \lim\limits_{n \to \infty}{2n} = \infty\]
		By the same argument, the $\inf\{s_k, s_{k+1}, s_{k+2}, \ldots\}$ is dominated by the odd terms of $s_n$, which is split between two cases: $s_k = \frac{1}{k} + 1$ and $s_k = \frac{1}{k} - 1$. Thus
		\[
			\varliminf \{s_n\} = 
			\lim\limits_{n\to\infty}{\frac{1}{4n+3}-1}
			= -1
		\]
	
	\item{$s_n = f(n)$, where $f \colon \mathbb{N} \to \mathbb{Q} \cap (0,1)$
		is a bijection}	\\\\
		Let $\epsilon > 0$ and $\overline{S} = \{s_n \colon 1 - s_n < \epsilon\}$ \\
		By Math340 results we know that there exists an infinite number of rationals between any two distinct reals (in this case 1 and $1 - \epsilon$). Therefore $\overline{S}$ is a countably infinite set, and $\overline{S} \setminus \{s_1, s_2, \ldots, s_{k-1}\}$ yields a similarly infinite set. From this,
		\[\varlimsup \{s_n\} = \sup{\overline{S}} = 1\]
		Similarly, with $\underline{S} = \{s_n \colon s_n < \epsilon\}$ we have 
		\[\varliminf \{s_n\} = \inf{\underline{S}} = 0\]
		
	\end{enumerate}

\item[\bf{Problem 2}] Give an example of a sequence $\{s_n\}$ and a real number $s$ such that $s_n < s$ for all $n$, but $\varlimsup s_n \geqslant s$ \\\\
	Let $s_n = 1-\frac{1}{n}$ and $s = 1$, then $\forall n, s_n < s$ and $\varlimsup s_n = s$
	
\item[\bf{Problem 3}] Let \{$a_n$\} and \{$b_n$\} be bounded sequences of nonnegative real numbers.
	\begin{enumerate}
	\item $\varlimsup(a_n b_n) \leqslant (\varlimsup a_n)(\varlimsup b_n)$ \\
	\item[Proof:]
	
	Since $a_n, b_n$ are bounded and nonnegative, let $A,B \in \mathbb{R}^+ \cup  \{0\}$ such that $\varlimsup \{a_n\} = A$ and $\varlimsup \{b_n\} = B$ \\
	By definition, $\forall n, a_n \leqslant A$ and $b_n \leqslant B$, so $a_n b_n \leqslant AB$ \\
	$AB$ is thus an upper bound on the sequence $\{a_n b_n\}$ \\ 
	Therefore $\varlimsup \{a_n b_n\} \leqslant AB$	
	\[\blacksquare\]
	\end{enumerate}
	
\item[\bf{Problem 4}] The series $\sum\limits_{k=1}^{\infty} \frac{1}{\sqrt{k}}$ diverges
	\begin{enumerate}
	\item[Proof:] Let $s_n = \sum\limits_{k=1}^{n} \frac{1}{\sqrt{k}}$ \\
	Then $s_n = 1 + \frac{1}{\sqrt{2}} + \cdots + \frac{1}{\sqrt{n}}$ \\
	Since $\frac{1}{\sqrt{k}}$ is monotone decreasing, we have
	\[
		s_n \geqslant 
		\frac{1}{\sqrt{n}} + \frac{1}{\sqrt{n}} + \cdots \frac{1}{\sqrt{n}}
		= \frac{n}{\sqrt{n}} = \sqrt{n}
	\]
	So for all $n \geqslant 1$, the partial sum $s_n \geqslant \sqrt{n}$ \\
	Therefore, $\varlimsup\{s_n\} \geqslant \varlimsup\{\sqrt{n}\} = \infty$ 
	\[\blacksquare\]
	\end{enumerate}

\end{enumerate}

\end{document}
