\documentclass[a4paper,12pt]{report}

\usepackage{amsmath}
\usepackage{amssymb}
\usepackage{amsfonts}
\usepackage{graphicx}

%Custom command to shorten mathbb for basic number sets
\newcommand{\bb}[1]{\mathbb{#1}}

%Custom command to create a subscript naught
\newcommand{\naught}[1]{{#1}_{\circ}}

%Custom command for defining problem{#}{theorem}proof:{text}blacksquare
\newcommand{\proof}[3]{
	\begin{enumerate}
		\item[\bf{Problem #1}] #2
		\begin{enumerate}
			\item[\textbf{Proof:}]
			#3
			\begin{flushright}
				$\blacksquare$
			\end{flushright}
		\end{enumerate}
	\end{enumerate}
}

%Custom command for defining problem{prob#}{text}
\newcommand{\problem}[3]{
	\begin{enumerate}
		\item[\bf{Problem #1}] #2 
		#3
	\end{enumerate}
}

%Custom command for defining {item#}{header}proof:{text}blacksquare
%does require \b{enum} ... \e{enum} or \problem
\newcommand{\subproof}[3]{
	\item[#1] #2
	\item[\bf{Proof:}] 
	#3 
	\begin{flushright}
		$\blacksquare$
	\end{flushright}
}

\begin{document}

\title{Math 440 -- Real Analysis II \\ \vspace{7px} \large{Exam 1}}
\author{Amandeep Gill}
\maketitle

\problem{1}{
	Given the series $\sum\limits_{k=1}^{\infty}p^{-k}k^p$, for what values of $p$ does it:
}{
	\subproof{(a)}{
		diverges when $-1 < p \leqslant 1$
	}{
		If $p = 0$ then $p^{-k}k^p = \frac{1}{0}$, which is unbounded and therefore divergent. \\
		If $p = 1$ then $p^{-k}k^p = k$, which is unbounded and therefore divergent. \\
		If $0 < p < 1$ then $p = 1/r$ for some $r > 1$ and $p^{-k}k^p = r^k \sqrt[r]{k}$, which is unbounded and therefore divergent. \\
		If $-1 < p < 0$ then $p = -1/r$ for some $r > 1$ and $p^{-k}k^p = (-1)^k \frac{r^k}{\sqrt[r]{k}}$. Since $r^k$ grows without bounds, and for all $r$, there exists some $\naught{k} \in \bb{N}$ such that $r^{\naught{k}} > \sqrt[r]{\naught{k}}$, $(-1)^{k}\frac{r^{k}}{k^p}$ is unbounded and therefore divergent.
	}
	
	\subproof{(b)}{
		converge conditionally when $p = -1$
	}{
		If $p = -1$ then $p^{-k}k^p = \frac{(-1)^k}{k}$, which is the oscillating harmonic series, and is conditionally convergent.
	}
	
	\subproof{(c)}{
		converge absolutely when $|p| > 1$
	}{
		If $p < -1$ then $p^{-k}k^p = \frac{(-1)^k}{p^k k^p}$. Since $\left|\frac{(-1)^k}{p^k k^p}\right| = \frac{1}{p^k k^p} \leqslant \frac{1}{k^p}$, the series $\sum\limits_{k=1}^{\infty}\frac{(-1)^k}{p^k k^p}$ converges absolutely by the Comparison Test against a P-series where $P > 1$.
		
		If $p > 1$ then using the Ratio Test leads to $\lim\limits_{k \to \infty}\frac{p^{-k-1}(k+1)^p}{p^k k^p} = \frac{1}{p} < 1$, meaning the series converges. Because $|p^k k^p| = p^k k^p$, the series is absolutely convergent.
	}
}

\pagebreak

\proof{2}{
	If $\{a_n\}$ is a sequence with $a_n > 0$ for all $n$ then $\sum\limits_{n=1}^{\infty}a_n$ converges if and only if ${\sum\limits_{n=1}^{\infty}}\frac{a_n}{1 + a_n}$ converges.
}{
	Let $\{a_n\}$ be a sequence with $a_n > 0$ for all $n$
	\item[$(\implies)$]
		Assume $\sum\limits_{n=1}^{\infty}a_n$ converges. Since $a_n > 0$, $a_n + 1 > 1$ which means that $\frac{a_n}{1 + a_n} < a_n$ for all $n$. Therefore, by the Comparison Test, ${\sum\limits_{n=1}^{\infty}}\frac{a_n}{1 + a_n}$ is a convergent series.
	\item[$(\impliedby)$] 
		Assume ${\sum\limits_{n=1}^{\infty}}\frac{a_n}{1 + a_n}$ converges. This implies that $\lim\limits_{n \to \infty}\frac{a_n}{1 + a_n} = 0$, so $\lim\limits_{n \to \infty} 1 - \frac{1}{1 + a_n} = 0$, which means $\lim\limits_{n \to \infty}a_n = 0$. $\{a_n\}$ is therefore a bounded sequence so there exists $m \in \bb{R^+}$ such that $m = \max\{a_n\}$. Then for all $n$, $\frac{a_n}{1 + m} \leqslant \frac{a_n}{1 + a_n}$ and by the Comparison Test $\sum\limits_{n=1}^{\infty}\frac{a_n}{1 + m}$ converges, and therefore so does $\sum\limits_{n=1}^{\infty}a_n$.
}

%\pagebreak

\problem{3}{
	Let $\{a_n\}$ be a sequence $a_n \neq 0$ for all $n$.
}{
	\subproof{(a)}{
		Disprove: if $\left|\frac{a_{n+1}}{a_n}\right| < 1$ for all $n$ then $\sum\limits_{n=1}^{\infty}a_n$ converges.
	}{
		Let $a_n = \frac{1}{n}$. For all $n$, $\left|\frac{a_{n+1}}{a_n}\right| = \left|\frac{n}{n+1}\right| < 1$, however $\sum\limits_{n=1}^{\infty}a_n$ is the harmonic series and therefore does not converge.
	}
	
	\subproof{(b)}{
		If there exists a convergent subsequence of $\{b_n\} = \left\{\sqrt[n]{|a_n|}\right\}_{n=1}^{\infty}$ whose limit is strictly greater than 1 then the series $\sum\limits_{n=1}^{\infty}a_n$ diverges
	}{
		Let $\{b_n\}$ have a subsequence that converges to some $b > 1$. Then $\limsup\limits_{n \to \infty}\{b_n\} \geqslant b$ since for all $\naught{n} \in \bb{N}$ and $\epsilon > 0$ there is an $n > \naught{n}$ such that $|b_n - b| < \epsilon$. Thus $\{b_n\}$ either diverges or converges to a number greater than 1, and the series diverges by the Root Test.
	}
}

\pagebreak

\problem{4}{
	For each $n \in \bb{N}$ let $f_n : [0,1] \to \bb{R}$ be defined as \\
	\begin{displaymath}
		f_n(x) = \left\{
			\begin{array}{lr}
				nx & \text{if } 0 \leqslant x \leqslant \frac{1}{n} \\
				2 - nx & \text{if } \frac{1}{n} < x \leqslant \frac{2}{n} \\
				0 & \text{if } \frac{2}{n} < x \leqslant 1
			\end{array}
		\right.
	\end{displaymath}
}{
	\item[(a)] 
		Plot of $f_n$ for $n = 2, 3, 4$ -- green, blue, and orange respectively: \\
		\includegraphics[scale=.25]{fn_plot.png}
		
	\subproof{(b)}{
		$\{f_n\}$ converges pointwise but not uniformly to some function $f$.
	}{
		Let $x \in [0,1]$ be given. If $x = 0$ then $\lim\limits_{n \to \infty}f_n=\lim\limits_{n \to \infty}nx=0$. Otherwise, if $0 < x \leqslant 1$ then there exists some $\naught{n} \in \bb{N}$ such that $\frac{2}{\naught{n}} < x$ and so for all $n > \naught{n}$, $f_n(x) = 0$. Therefore $\{f_n\}$ converges pointwise to $f(x)=0$. However letting $\epsilon = \frac{1}{2}$, $n>N$, and $x = \frac{1}{2n}$ for some $N \in \bb{N}$ gives $|f_n(x)-f(x)| = |nx-0| = \left|\frac{n}{2n}\right| = \frac{1}{2} = \epsilon$, showing that $\{f_n\}$ is not uniformly convergent.
	}
	
	\item[(c)] 
		$f_n$ is continuous for all $n$ on [0,1], and the same holds for $f$.
	
	\item[(d)] 
		Parts (b) and (c) do not violate Corollary 8.3.2 because it does not address functions that do not converge uniformly.
}

\pagebreak

\problem{5}{}{
	\subproof{(a)}{
		$\frac{1}{1-x} = \sum\limits_{k=0}^{\infty}x^k$ when $x \in (-1,1)$
	}{
		Let $s_n = \sum\limits_{k=0}^{n}x^k$, then $s_n = 1 + x + x^2 + \cdots + x^n = \frac{1}{1-x} - \frac{x^n}{1-x}$. Since $|x| < 1$, $\lim\limits_{n \to \infty}s_n = \lim\limits_{n \to \infty}\frac{1}{1-x} - \frac{x^n}{1-x} = \frac{1}{1-x}$.
	}
	
	\subproof{(b)}{
		Evaluate $f(x)=\int_{0}^{1/2}\frac{1}{1+x^6}dx$ as a power series.
	}{
		By part (a), $\frac{1}{1+x^6} = \sum\limits_{k=0}^{\infty}(x^6)^k$ since $x^6 \in (-1,1)$ for all $x \in \left[0,\frac{1}{2}\right]$. Substituting the series in the integral gives $\int_{0}^{1/2}\sum\limits_{k=0}^{\infty}(x^6)^k dx = \sum\limits_{k=0}^{\infty}\int_{0}^{1/2}(x^6)^k dx$ because the Power Series is uniformly convergent on its domain. This gives $f(x) = \sum\limits_{k=0}^{\infty}\frac{(x^6)^{k+1}}{k+1}$.
	}
}

\pagebreak

\problem{6}{
	Using $f(x) = \sum\limits_{k=1}^{\infty}f_{k}(x) = \sum\limits_{k=1}^{\infty}\frac{1}{1 + k^2 x}$ prove:
}{
	\subproof{(a)}{
		That $f(x)$ converges uniformly on $x \in [a,\infty)$ for all $a>0$
	}{
		Since $a \leqslant x$ for all $x \in [a,\infty)$, $\frac{1}{1+k^2 x} \leqslant \frac{1}{1+k^2 a} < \frac{1}{k^2 a}$ for all $k \in \bb{N}$. Since $\sum\limits_{k=1}^{\infty}\frac{1}{k^2 a}$ is a convergent P-series, $f(x)$ converges uniformly on $[a,\infty)$ by the Weierstrass M-test.
	}
	
	\subproof{(b)}{
		That $f(x)$ does not converge uniformly on $x \in (0,\infty)$
	}{
		
		Let $\epsilon = \frac{1}{2}$, $\naught{k},m,n \in \bb{N}$ such that $\naught{k} < m < n$, and $x=\frac{1}{n^2}$. Then by the Cauchy Criterion, $\left|f_n(x) - f_m(x)\right| = \left|\sum\limits_{k=1}^{n}\frac{1}{1 + k^2 x} - \sum\limits_{k=1}^{m}\frac{1}{1 + k^2 x}\right| = \sum\limits_{k=m+1}^{n}\frac{1}{1 + k^2 x} \geqslant \sum\limits_{k=m+1}^{n}\frac{1}{1 + n^2 x} = \sum\limits_{k=m+1}^{n}\frac{1}{2} \geqslant \frac{1}{2} = \epsilon$, and the series is therefore not uniformly convergent on $(0,\infty)$.
	}
}

\pagebreak

\problem{7}{
	Let $\{a_n\} = \left\{\frac{1}{2n-1}\right\}_{n \in \bb{N}} = 1,\frac{1}{3},\frac{1}{5},\frac{1}{7},\ldots$ and $\{b_n\} = \left\{\frac{1}{2n}\right\}_{n \in \bb{N}} = \frac{1}{2},\frac{1}{4},\frac{1}{6},\ldots$
}{
	\subproof{(a)}{
		The series $\sum\limits_{k=1}^{\infty}\frac{(-1)^{k+1}}{k} = a_1 - b_1 + a_2 - b_2 + \cdots$ converges.
	}{
		The series converges by the Alternating Series Test, and since it is the same term-by-term sequence of numbers as the series $\sum\limits_{n=1}^{\infty}(a_n - b_n)$, it follows that $a_1 - b_1 + a_2 - b_2 + \cdots$ converges as well.
	}
	
	\subproof{(b)}{
		For each integer $j \geqslant 1$ let $n_j = 2^j - 1$, $a_{1+n_{j-1}} + a_{2+n_{j-1}} + \cdots + a_{n_{j}} \geqslant \frac{1}{4}$
	}{
		By assumption, $a_{1+n_{j-1}} = \frac{1}{1+2(2^{j-1}-1)-1} = \frac{1}{2^j-1}$ and $a_{n_{j}} = \frac{1}{2(2^{j}-1)-1} = \frac{1}{2^{j+1}-3}$. The distance between the denominators is equal to $(2^{j+1}-3)-(2^j-1) + 1 = 2(2^{j-1})$, giving $2^{j-1}$ terms in the sum $s_j$. Since each term in $s_{j_i} \geqslant \frac{1}{2^{j+1}-3} \geqslant \frac{1}{2^{j+1}}$, then $s_j \geqslant \frac{2^{j-1}}{2^{j+1}} = \frac{1}{4}$.
	}
	
	\subproof{(c)}{
		Let the rearrangement of $\sum\limits_{k=1}^{\infty}\frac{(-1)^{k+1}}{k}$ be given as $\sum\limits_{k=1}^{\infty}(s_k - b_k)$
	}{
		Let $\sum\limits_{k=1}^{\infty}c_k = \sum\limits_{k=1}^{\infty}(s_k - b_k)$, then the first four terms are:\\
		$c_1 = a_1 - b_1$ \\
		$c_2 = a_2 + a_3 - b_2$ \\
		$c_3 = a_4 + a_5 + a_6 + a_7 - b_3$ \\
		$c_4 = a_8 + a_9 + a_10 + a_11 + a_12 + a_13 + a_14 + a_15 - b_4$
	}
	
	\subproof{(d)}{
		The rearrangement $\sum\limits_{k=1}^{\infty}c_k$ diverges
	}{
		$\lim\limits_{k \to \infty}c_k = \lim\limits_{k \to \infty}s_k - b_k$. Since $\lim\limits_{k \to \infty}b_k = 0$, and $s_k \geqslant \frac{1}{4}$ by previous work, so $\lim\limits_{k \to \infty}c_k \geqslant \frac{1}{4} \neq 0$. Thus the series fails the $n^{\text{th}}$-Term Test and is therefore divergent.
	}
}

\end{document}