\documentclass[a4paper,12pt]{report}

\usepackage{amsmath}
\usepackage{amssymb}
\usepackage{amsfonts}

%Custom command to shorten mathbb for basic number sets
\newcommand{\bb}[1]{\mathbb{#1}}

%Custom command for defining problem{#}{theorem}proof:{text}blacksquare
\newcommand{\proof}[3]{
	\begin{enumerate}
		\item[\bf{Problem #1}] #2
		\begin{enumerate}
			\item[\underline{Proof:}]
			#3
			\begin{flushright}
				$\blacksquare$
			\end{flushright}
		\end{enumerate}
	\end{enumerate}
}

%Custom command for defining problem{prob#}{text}
\newcommand{\problem}[3]{
	\begin{enumerate}
		\item[\bf{Problem #1}] #2 
		#3
	\end{enumerate}
}

%Custom command for defining {item#}{header}proof:{text}blacksquare
%does require \b{enum} ... \e{enum} or \problem
\newcommand{\subproof}[3]{
	\item[#1] #2
	\item[\bf{Proof:}] 
	#3 
	\begin{flushright}
		$\blacksquare$
	\end{flushright}
}

\begin{document}

\title{Math 440 -- Real Analysis II \\ \vspace{7px} \large{Homework 2}}
\author{Amandeep Gill}
\maketitle

\problem{1}{Test the following series for convergence.}{
	\subproof{(a)}{
		$\sum\limits_{k=1}^{\infty}{\frac{k+1}{2k^3-k}}$ Does converge
	}{
		Let $a_k = \frac{1}{k^2}$, then 
		$\lim\limits_{k \to \infty}{\frac{s_k}{a_k}} = 
		\lim\limits_{k \to \infty}{\frac{(k+1)(k^2)}{2k^3-k}} = \frac{1}{2}$ \\
		Since $\frac{1}{2} \in \mathbb{R^+}$, $\sum\limits_{k=1}^{\infty}{\frac{k+1}{2k^3-k}}$ converges by the Limit Comparison Test.
	}
	
	\subproof{(b)}{
		$\sum\limits_{n=1}^{\infty}{\frac{p^n}{n!}}$, $p \in \mathbb{R}$ Does converge
	}{
		Using the ratio test, let $L = \lim\limits_{k \to \infty}(\frac{p^{k+1}}{(k+1)!})(\frac{k!}{p^{k}}) = \lim\limits_{k \to \infty} \frac{p}{k + 1} = 0$ \\
		Since $L < 1$, the series $\sum\limits_{n=1}^{\infty}{\frac{p^n}{n!}}$ does converge.
	}
}

\problem{2}{
		Suppose $a_k \geqslant 0$ for all $k \in \bb{N}$ and $\sum a_k < \infty$. Prove that the following converge or give an example where the series does not.
	}{
	\subproof{(a)}{
		$\sum\limits_{k=1}^{\infty}{a_k}^2$ is guaranteed to converge
	}{
		By assumption $a_k \geqslant 0$ for all $k \in \bb{N}$, so $a_k = |a_k|$ and $\sum |a_k| = \sum a_k$. The series is thus absolutely convergent, so for some $m = max\{a_1, a_2, \ldots\}$, $m\sum a_k = \sum m a_k < \infty$. Since $a_k \leqslant m,\ {a_k}^2 \leqslant m a_k$, the series $\sum\limits_{k=1}^{\infty}{a_k}^2$ converges by the Weierstrass M-test.
	}
	
	\subproof{(b)}{
		$\sum\limits_{k=1}^{\infty}\frac{{a_k}^2}{1+a_k}$
	}{
		pf
	}
}

\end{document}
