\documentclass[a4paper,12pt]{report}

\usepackage{../ssumath}

\begin{document}
	
\mktitle{Math 440 -- Real Analysis II}{Homework 10}{Amandeep Gill}

\proof{58}{
	Let $f$ and $g$ be nonnegative, real-valued functions defined on a measurable set $A$ with $n \in \bb{N}$, then for all $x \in A$ 
	\[\min\{f(x) + g(x), n\} \leqslant \min\{f(x), n\} + \min\{g(x), n\} \leqslant \min\{f(x) + g(x), 2n\}\]
}{
	Let $f,g : A \to \bb{R}^{0+}$
	\begin{enumerate}
		\case{let $f(x) > n$ and $g(x) > n$}: 
	
		Then the inequality holds trivially.
		
		\case{wolog let $f(x) > n$ and $g(x) \leqslant n$}:
		
		Then $n < n + g(x) \leqslant f(x) + g(x)$ and $2n$, so the inequality holds.
		
		\case{let $f(x) \leqslant n$, $g(x) \leqslant n$, and $f(x) + g(x) > n$}:
		
		Then $f(x) + g(x) \leqslant 2n$, so the inequality holds.

		\case{let $f(x) + g(x) \leqslant n$}:
		
		Then the inequality holds trivially.
	\end{enumerate}
}

\proof{59}{
	If $f : (0, \infty) \to \bb{R}$ such that for each $n \in \bb{N}$, $f(x) = (-1/2)^n$ for $n-1 \leqslant x < n$, then $f$ is Lebesgue integrable and $\int\limits_{(0,\infty)} f d\lambda = -\frac{1}{3}$.
}{
	If $\lfloor x \rfloor$ is even then $f(x) \leqslant 0$, and if $\lfloor x \rfloor$ is odd then $f(x) < 0$. So let $f^-(x) = (1/2)^{2n-1}$ for $2n - 2 \leqslant x < 2n - 1$ and let $f^+(x) = (1/2)^{2n}$ for $2n - 1 \leqslant x < 2n$. Thus $f^+$ and $f^-$ are nonnegative and bounded, with $f(x) = f^+(x) - f^-(x)$. Additionally, both functions are continuous on a countable union of disjoint intervals, so by Theorems 10.5.5 and 10.4.5 the functions $f^-$ and $f^+$ are measurable. Hence, by Theorem 10.7.1, $f^-$ and $f^+$ are Lebesgue integrable, and by 10.7.4 $f$ is as well.
	
	Therefore:
	\longeq{
		\eqstep{\int\limits_{(0,\infty)} f d\lambda}{\int\limits_{(0,\infty)} f^+ d\lambda - \int\limits_{(0,\infty)} f^- d\lambda}
		\eqstep{}{\lim\limits_{n \to \infty}\int\limits_{(0,n]} f^+ d\lambda - \lim\limits_{n \to \infty}\int\limits_{(0,n]} f^- d\lambda}
		\eqstep{}{\lim\limits_{n \to \infty}\sum\limits_{k=1}^{n}(\frac{1}{2^{2k}})\lambda[2k-2,2k-1) - \lim\limits_{n \to \infty}\sum\limits_{k=1}^{n}(\frac{1}{2^{2k-1}})\lambda[2k-1,2k)}
		\eqstep{}{\sum\limits_{k=1}^{\infty}(\frac{1}{2^{2k}}) - \sum\limits_{k=1}^{\infty}(\frac{1}{2^{2k-1}}) = -\frac{1}{3}}
	}
}

\problem{60}{
	Let $f,g : A \to \bb{R}^{0+}$ be measurable
}{
	\subproof{(b)}{
		If $A_1,A_2$ are measurable, disjoint subsets of $A$, then \[\int\limits_{A_1 \cup A_2}f d\lambda = \int\limits_{A_1}f d\lambda + \int\limits_{A_2}f d\lambda\]
	}{
		Let $f_1 = f\chi_{A_1}$ and $f_2 = f\chi_{A_2}$. Then $\int\limits_{A_1}f d\lambda = \int\limits_{A_1}f_1 d\lambda$ and $\int\limits_{A_2}f d\lambda = \int\limits_{A_2}f_2 d\lambda$, and since $f_1(x) = 0$ for all $x \not\in A_1$ and $f_2(x) = 0$ for all $x \not\in A_2$, $\int\limits_{A_1}f d\lambda = \int\limits_{A_1 \cup A_2}f_1 d\lambda$ and $\int\limits_{A_2}f d\lambda = \int\limits_{A_1 \cup A_2}f_2 d\lambda$. As $A_1,A_2$ are disjoint, $(f_1+f_2)(x) = f(x), \forall x \in A_1 \cup A_2$, thus
		\longeq{
			\eqstep{\int\limits_{A_1}f d\lambda + \int\limits_{A_2}f d\lambda}{\int\limits_{A_1 \cup A_2}f_1 d\lambda + \int\limits_{A_1 \cup A_2}f_2 d\lambda}
			\eqsteptx{}{\int\limits_{A_1 \cup A_2}(f_1 + f_2) d\lambda}{By Thm 10.7.4(a)}
			\eqstep{}{\int\limits_{A_1 \cup A_2}f d\lambda}
			
		}
	}
	
	\subproof{(c)}{
		If $f \leqslant g$ a.e. on $A$, then $\int\limits_{A}f d\lambda \leqslant \int\limits_{A}g d\lambda$ with equality if $f = g$ a.e.
	}{
		Let $E = \{x \in A : f(x) > g(x)\}$, $E_1 = \{x \in A : f(x) = g(x)\}$, and $E_2 = \{x \in A : f(x) < g(x)\}$. Because $\lambda(E) = 0$ by assumption, $\int\limits_{A}f d\lambda = \int\limits_{E_1 \cup E_2}f d\lambda$ by Theorem 10.7.4(b), and similarly for $g$. Let $h(x) = (g - f)(x)$ for all $x \in E_1 \cup E_2$. As $h$ is nonnegative, 
		\longeq{
			\eqstep{\int\limits_{A}g d\lambda}{\int\limits_{E_1 \cup E_2}g d\lambda}
			\eqstep{}{\int\limits_{E_1 \cup E_2}(f + h) d\lambda}
			\eqsteptx{}{\int\limits_{E_1 \cup E_2}f d\lambda + \int\limits_{E_1 \cup E_2}h d\lambda}{By Thm 10.7.4(a)}
			\eqsteptx{}{\int\limits_{A}f d\lambda + \int\limits_{E_1}h d\lambda + \int\limits_{E_2}h d\lambda}{By Thm 10.7.4(b)}
			\eqsteptx{}{\int\limits_{A}f d\lambda + \int\limits_{E_2}h d\lambda}{As $h(x) = 0,\ \forall x \in E_1$}
		}
		Therefore if $g = f$ a.e., then $\int\limits_{A}f d\lambda = \int\limits_{A}g d\lambda$, otherwise $\int\limits_{A}f d\lambda < \int\limits_{A}g d\lambda$.
	}
}

\proof{61}{
	Let $f$ be a nonnegative integrable function on $[a,b]$. For each $n \geqslant 0$, if $E_n = \{x : n \leqslant f(x) < n+1\}$ then $\sum\limits_{n=0}^{\infty}n\lambda(E_n) < \infty$.
}{
	Let $f_k$ be defined such that $f_k(x) = f(x)$ if $f(x) < k+1$ and $0$ otherwise, then $f_k(x) = f(x)$ for all $x \in E_k$ and $f_k(x) = 0$ for all $x \in E_k^c$ where $A_k = \bigcup\limits_{n=0}^{k}E_n$ and $A_k^c = [a,b] \backslash A_k$. By Theorem 10.7.4, $\int\limits_{[a,b]}f d\lambda = \int\limits_{A_k}f d\lambda + \int\limits_{A_k^c}f d\lambda$, and because $f$ is integrable, $\int\limits_{A_k^c}f d\lambda \geqslant 0$, and $f_k(x) = f(x)$ for all $x \in A_k$, $\int\limits_{A_k}f_k d\lambda \leqslant \int\limits_{[a,b]}f d\lambda < \infty$ for all $k \geqslant 0$. Since $f_k$ is bounded on $[a,b]$ and $\partn_k = \{A_k^c, E_0, E_1, \ldots, E_k\}$ partitions $[a,b]$, $\mathcal{L}_L(\partn_k, f_k) \leqslant \int\limits_{[a,b]}f_k d\lambda$ by Definition 10.6.3 for all $k$, and so $\mathcal{L}_L(\partn_k, f_k) = 0\lambda(A_k^c) + \sum\limits_{n=0}^{k}m_n\lambda(E_n)$. Because for all $n \leqslant k$, $n \leqslant m_n = \inf\{f_k(x)$ for all $x \in E_n\}$, $\sum\limits_{n=0}^{k}m_n\lambda(E_n) \leqslant \mathcal{L}_L(\partn_k, f_k)$. Therefore $\sum\limits_{n=0}^{k}m_n\lambda(E_n) \leqslant \int\limits_{[a,b]}f d\lambda < \infty$ for all $k \geqslant 0$, so $\sum\limits_{n=0}^{\infty}m_n\lambda(E_n) < \infty$.
}

\pagebreak

\proof{62}{
	Let $f$ be a nonnegative measurable function on a measurable set $A$. If $\int\limits_{A}f d\lambda = 0$ then $f = 0$ almost everywhere on $A$.
}{
	\begin{enumerate}
		\item For each $n \in \bb{N}$ let $f_n : A \to [0,\infty)$ be defined as $f_n(x) = \min\{f(x), n\}$, then $f_n(x) \leqslant f(x)$ for all $n \in \bb{N}$ and $x \in A$ since $\min\{f(x),n\} \leqslant f(x)$. 
	
		\item Let $n,m \in \bb{N}$ be given and $A_m = A \cap [-m,m]$ so that, by Theorem 10.7.4(b), $\int\limits_{A}f d\lambda = \int\limits_{A \cap A_M}f d\lambda + \int\limits_{A \backslash A_M}f d\lambda$. As the Lebesgue integral is nonnegative and $A \cap A_m = A_m$, $\int\limits_{A_m}f d\lambda = 0$. Hence $\int\limits_{A_m}f_n d\lambda = 0$ given that $f_n(x) \leqslant f(x)$ for all $x \in A_m$. 
	
		\item Let $E_{m,n} = \{x \in A_m : f_n(x) \not= 0\}$ such that $\int\limits_{A_m}f_n d\lambda = \int\limits_{E_{m,n}}f_n d\lambda + \int\limits_{A_m \backslash E_{m,n}}f_n d\lambda$. By previous argument $\int\limits_{E_{m,n}}f_n d\lambda = 0$, and by Definition 10.6.3 $\int\limits_{E_{m,n}}f_n d\lambda \geqslant i \lambda(E_m)$, where $i =  \inf\{f_n(x) : x \in E_m\}$. Therefore $i \lambda(E_{m,n}) = 0$, so $\lambda(E_{m,n}) = 0$ as $i \geqslant 0$. Because $n,m$ are arbitrary, $\lambda(E_{m,n}) = 0$ for all $m,n \in \bb{N}$
		
		\item Let $E = \{x \in A : f(x) \not= 0\}$. Then since $n > 0$ for all $n \in \bb{N}$ and $f(x) > 0$ for all $x \in E$, this gives $f_n(x) \not= 0$ for all $x \in E$ and all $n \in \bb{N}$. Additionally, for all fixed $x \in E$ there exists an $m \in \bb{N}$ such that $x < m$. Therefore, for every $x \in E$, $x \in E_{m,n}$ for some $m,n \in \bb{N}$ and thus $E \subset \bigcup\limits_{m=1}^{\infty}\bigcup\limits_{n=1}^{\infty}E_{m,n}$.
		
		\item By Theorem 10.4.5, $\lambda(E) \leqslant \lambda\left(\bigcup\limits_{m=1}^{\infty}\bigcup\limits_{n=1}^{\infty}E_{m,n}\right) \leqslant \sum\limits_{m=1}^{\infty} \sum\limits_{m=1}^{\infty} \lambda(E_{m,n})$. Since $\lambda(E_{m,n}) = 0$ for all $m,n \in \bb{N}$, $\lambda(E) = 0$, so $f = 0$ almost everywhere.
	\end{enumerate}
}

\end{document}












